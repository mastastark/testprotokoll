%! TeX program = lualatex
%---------------------------ALLGEMEINE IMPORTS-------------------------------------
\documentclass[12pt,english,ngerman]{scrartcl}
\input{./protokoll_template/template.latex/input/shared_preamble.tex}

% Kopfzeile
\ihead{WS22\\
	21.12.2022} \chead{\textsc{Stark} Matthias - 12004907 \\
	\textsc{Philipp} Maximilian - 11839611}
\ohead{FLAB 1 \\
	Rasterelektronenmikroskopie}
% Fußzeile


\begin{document}

\section{Grundlagen}

%(max. 1 Seite)


\section{Proben- und Geräteliste}


\section{Kennenlernen des REM}


Zunächst wird die Probe mit einer Pinzette vorsichtig auf den Probentisch platziert. Für ein effektives Arbeiten können 
mehrere Proben gleichzeitig auf die Probenbühne gesetzt und diese zwischen den einzelnen Messungen einfach weitergedreht
werden. Im Rahmen des Praktikums wird jedoch immer nur eine Probe eingelegt, um das Handling zu lernen und sicherzustellen,
dass die "z-Ebene", also die Höhe, immer richtig eingestellt ist.

Bei der Probe ist zu beachten, dass diese elektrisch leitfähig sein muss. Ist die Probe von sich aus schon leitfähig, wird
sie mit einem speziellen, leitenden Kohlenstoff-Band am Sockel befestigt. Handelt es sich um eine nicht leitende Probe, so 
muss die leitfähigkeit z.B. durch eine Platin Bedampfung gewährleistet werden, wie im \autoref{fig:probe} sichtbar.

\begin{figure}[H]
	\begin{center}
		\includegraphics[width =0.5\textwidth]{./figures/probe.png}
	\end{center}
	\caption[Bedampfte, organische Probe]
	{Bedampfte, organische Probe \cite{sein_foto}}
    \label{fig:probe}
\end{figure}

Nach dem Einlegen der Probe wird ein Vakuum erzeugt, welches für den Betrieb des Elektronenmikroskops notwendig ist, was 
\SI{162(1)}{\s}, also keine \SI{3}{\min} dauert.

Nun wird das aufgezeichnete Bild im verwendeten Computerprogramm sichtbar. Durch Bewegung mit der Computermaus kann 
der entsprechende Bereich ausgewählt und die Vergrößerung eingestellt werden. Auch kann über den entsprechenden Knopf die 
Schärfe, sowie der Kontrast, variiert werden, um ein möglichst gut aufgelöstes Bild zu erreichen. Man muss sich bewusst 
sein, dass wie bei allen
optischen Aufbauten, gewisse Abbildungsfehler vorliegen. Für eine genauere Erklärung hierzu, sein auf \cite{unterlagen} 
verwiesen. Der Astigmatismus durch eine entsprechende Anpassung im jeweiligen Menüpunkt großteils behoben werden.


%todo gets vlt dass wir jeweils 2 Bilder nebeneinander kriegen? 

Im folgenden ist eine Auswahl der erzeugten Bilder angeführt. In \autoref{fig:auge} und \autoref{fig:auge2} sind Aufnahmen
des Facettenauges sichtbar. In \autoref{fig:flugel} sieht man die Struktur des Flügels und in \autoref{fig:damage} ist eine
komisch geformte Struktur sichtbar, die auf einen Fehler in der Bedampfungsschicht zurückzuführen ist.

\begin{figure}[H]
	\begin{center}
		\includegraphics[width =0.5\textwidth]{./figures/auge.png}
	\end{center}
	\caption{Facettenauge}
    \label{fig:auge}
\end{figure}

\begin{figure}[H]
	\begin{center}
		\includegraphics[width =0.5\textwidth]{./figures/auge2.png}
	\end{center}
	\caption{stärker vergrößertes Facettenauge}
    \label{fig:auge2}
\end{figure}

\begin{figure}[H]
	\begin{center}
		\includegraphics[width =0.5\textwidth]{./figures/flugel.png}
	\end{center}
	\caption{Struktur des Flügels}
    \label{fig:flugel}
\end{figure}

\begin{figure}[H]
	\begin{center}
		\includegraphics[width =0.5\textwidth]{./figures/damage.png}
	\end{center}
	\caption{Fehler in Bedampfungsschicht}
    \label{fig:damage}
\end{figure}


\section{Polypropylen-Gewebe}

\subsection{Vergleich „beschichtet“ und „unbeschichtet“}


\subsection{Variation der Beschleunigungsspannung}


\section{Keramik}


\subsection{Vergleich SE- und BSE-Abbildung}


\subsection{Bestimmung der Schichtdicke}


\section{Qualitative EDX-Analyse}


\section{Quantitative EDX-Analyse}


\section{Zusammenfassung}


\section{Anhang}


%Vorbereitungsunterlagen, Daten über Münzen



\newpage

\printbibliography

\end{document}